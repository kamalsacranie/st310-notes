% Importing our necessary packages
\usepackage{booktabs}
% Footnotes fixed to bottom
\usepackage[bottom]{footmisc}
\usepackage{multicol}
\usepackage[edges]{forest}
\usepackage{fancyhdr}
\usepackage{titlesec}
\usepackage{cancel}
% Setting our text to not bet justified across the whole page and no hyphen
\usepackage[document]{ragged2e}
\usepackage[none]{hyphenat}

% Importing our tikz
\usetikzlibrary{trees}
\usetikzlibrary{backgrounds}

% Setting our mono code to use fira code
\usepackage{fontspec}
\setmonofont[
	Contextuals={Alternate}
]{Hasklug Nerd Font Mono}

% Setting up our hyperlinks
\hypersetup{colorlinks = true, linkcolor = blue, urlcolor = blue}

% Changing chaper text
\renewcommand{\chaptername}{Chapter}
% Chaning padding between footnote line and text
\addtolength{\skip\footins}{1em}
% % Change space between bullets
% \renewcommand{\tightlist}{
% \setlength{\itemsep}{0.4em}
% \setlength{\topsep}{1cm}
% \setlength{\partopsep}{1cm}
% }
% Decreasing space before chapter title
\titleformat{\chapter}[display]{\normalfont\Large\bfseries}{\chaptertitlename\ \thechapter}{0pt}{\huge}
\titlespacing*{\chapter}{10pt}{0pt}{20pt}

% Making the repositioning of our images more forgiving
\renewcommand{\topfraction}{.85}
\renewcommand{\bottomfraction}{.7}
\renewcommand{\textfraction}{.15}
\renewcommand{\floatpagefraction}{.66}
\setcounter{topnumber}{3}
\setcounter{bottomnumber}{3}
\setcounter{totalnumber}{4}

% Creating shaded indented box for quote
\usepackage{xcolor}
\usepackage[framemethod=TikZ]{mdframed}

\colorlet{quoteshadecolor}{yellow!10!white}
\renewenvironment{quote}{
	\bigskip\begin{mdframed}[
			skipabove=\topskip,
			skipbelow=\topskip,
			backgroundcolor=quoteshadecolor,
			leftmargin=0.5cm,
			rightmargin=0.5cm,
			topline=false,
			rightline=false,
			bottomline=false,
			nobreak=true,
		]\itshape%itemshape is for italics
		}{
	\end{mdframed}
}


% Changing the background color of our code blocks
% \colorlet{shadecolor}{magenta!20!white}
% \definecolor{code}{RGB}{178,178,178}
% Defining our shadecolor (used fo shading code blocks usually)
\definecolor{code}{RGB}{1,22,80} % This is how you define a color in latex
\colorlet{shadecolor}{code} % redefining our shadecolor to be code color
% Using tcolorbox to make a background box
\usepackage[many]{tcolorbox}
\tcbset{enhanced,colback=red!5!white,
	colframe=red!75!black,fonttitle=\bfseries}
% Renewing the shaded environemnt with new mdframed box
\renewenvironment{Shaded}{
	\bigskip
	\begin{tcolorbox}[drop fuzzy midday shadow]
		\begin{mdframed}[
				skipabove=\topskip*2,
				outerlinewidth= 0,
				linewidth=0pt,
				roundcorner= 3pt,
				backgroundcolor= shadecolor,
				outerlinecolor= shadecolor,
				innertopmargin= \topskip,
				innerbottommargin=\topskip,
				leftmargin=-0.8cm,
				rightmargin=-0.8cm
			]}{
		\end{mdframed}
	\end{tcolorbox}
	\smallskip
}

% fancyhdr for header and footer
\usepackage{fancyhdr}

\pagestyle{fancy}
\fancyhf{}
\fancyhead[LE,RO]{Kamal Sacranie}
\fancyhead[RE,LO]{Chapter \thechapter}
\fancyfoot[C]{\thepage}

\renewcommand{\headrulewidth}{2pt}
\renewcommand{\footrulewidth}{0pt}
