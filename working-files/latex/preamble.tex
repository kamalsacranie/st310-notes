% Importing our necessary packages
% Making headings start at same place at first lettter
\usepackage{titlesec}
\titlelabel{\llap{\thetitle\quad}}

\usepackage{booktabs}
% To color table cells
\usepackage{colortbl}
% Footnotes fixed to bottom
\usepackage[bottom]{footmisc}
\usepackage{multicol}
\usepackage[edges]{forest}
\usepackage{fancyhdr}
\usepackage{cancel}
% Allows us to have subfigures
\usepackage{subfig}
% Setting our text to not bet justified across the whole page and no hyphen
\usepackage[document]{ragged2e}
\usepackage[none]{hyphenat}
% Used ot keep floats in example boxes
\usepackage{float}

% Importing our tikz
\usetikzlibrary{trees}
\usetikzlibrary{backgrounds}

% Setting our mono code to Hasklug (must be install on system)
\usepackage{fontspec}
\setmonofont[
	Contextuals={Alternate}
]{Hasklug Nerd Font Mono}

% Setting up our hyperlinks
\hypersetup{colorlinks = true, linkcolor = blue, urlcolor = blue}

% Changing chaper text
\renewcommand{\chaptername}{Chapter}
% Chaning padding between footnote line and text
\addtolength{\skip\footins}{1em}
% % Change space between bullets
% \renewcommand{\tightlist}{
% \setlength{\itemsep}{0.4em}
% \setlength{\topsep}{1cm}
% \setlength{\partopsep}{1cm}
% }
% Decreasing space before chapter title
\titleformat{\chapter}[display]{\normalfont\Large\bfseries}{\chaptertitlename\ \thechapter}{0pt}{\huge}
\titlespacing*{\chapter}{10pt}{0pt}{20pt}

% Making the repositioning of our images more forgiving
\renewcommand{\topfraction}{.85}
\renewcommand{\bottomfraction}{.7}
\renewcommand{\textfraction}{.15}
\renewcommand{\floatpagefraction}{.66}
\setcounter{topnumber}{3}
\setcounter{bottomnumber}{3}
\setcounter{totalnumber}{4}

% Creating shaded indented box for quote
\usepackage{xcolor}
\usepackage[framemethod=TikZ]{mdframed}

\colorlet{quoteshadecolor}{yellow!10!white}
\renewenvironment{quote}{
	\bigskip\begin{mdframed}[
			skipabove=\topskip,
			skipbelow=\topskip,
			backgroundcolor=quoteshadecolor,
			leftmargin=0.5cm,
			rightmargin=0.5cm,
			topline=false,
			rightline=false,
			bottomline=false,
			nobreak=true,
		]\itshape%itemshape is for italics
		}{
	\end{mdframed}
}


% Defining our shadecolor (used fo shading code blocks usually)
\definecolor{code}{RGB}{1,22,80} % This is how you define a color in latex
\colorlet{shadecolor}{code} % redefining our shadecolor to be code color
% Using tcolorbox to make a background box
\usepackage[many, listings]{tcolorbox}
\newtcolorbox{codeboxback}{
	enhanced,
	colback=white!50!black,
	colframe=white!50!black,
	fonttitle=\bfseries,
	breakable,
	drop fuzzy midday shadow=black!30!white,
}
% Renewing the shaded environemnt with new mdframed box
\renewenvironment{Shaded}{
	\bigskip
	% \begin{codeboxback}%[drop fuzzy midday shadow]
	\begin{mdframed}[
			skipabove=\topskip*2,
			outerlinewidth= 0,
			linewidth=0pt,
			roundcorner= 3pt,
			backgroundcolor= shadecolor,
			outerlinecolor= shadecolor,
			innertopmargin= \topskip,
			innerbottommargin=\topskip,
			leftmargin=-0.4cm,
			rightmargin=-0.4cm
		]}{
	\end{mdframed}
	% \end{codeboxback}
	\smallskip
}

% fancyhdr for header and footer
\usepackage{fancyhdr}

\pagestyle{fancy}
\fancyhf{}
\fancyhead[R]{Kamal Sacranie}
\fancyhead[L]{Chapter \thechapter}
\fancyfoot[C]{\thepage}
% Changing header rule style
\renewcommand{\headrulewidth}{2pt}
\renewcommand{\footrulewidth}{0pt}

%%%%% THE SOLUTION TO WRAPPING ENVS!!!
% Styling ocde output
\AddToHook{env/verbatim/before}{\begin{mdframed}[
			skipabove=\topskip*2,
			outerlinewidth= 0,
			linewidth=0pt,
			backgroundcolor= black!10!white,
			outerlinecolor= black!40!white,
			roundcorner=5pt,
			innertopmargin= \topskip,
			innerbottommargin=\topskip,
		]}
		\AddToHook{env/verbatim/after}{\end{mdframed}}

% Styling the example box
\newtcolorbox{examplebox}[2][]{
	colback=green!20!white,
	colframe=green!20!black,
	coltitle=white,
	fonttitle=\bfseries,
	colbacktitle=green!20!black,
	enhanced,
	breakable,
	grow to left by=0.5cm,
	grow to right by=0.5cm,
	left=0.7cm,
	right=0.5cm,
	attach boxed title to top center={yshift=-2mm},
	title={#2},#1
}
% Makes the examplebox handle footnotes correctly
\makesavenoteenv{examplebox}
% Enclosing our example envs in a tcolorbox
\AddToHook{env/example/before}{\begin{examplebox}{Example}}
		\AddToHook{env/example/after}{\end{examplebox}}
